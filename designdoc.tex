\documentclass[letter,10pt]{article}
%\usepackage{fullpage}

\author{Devin Quirozoliver}
\title{Programming Project 2: Threaded Buffer Using PThreads\\Design Document and Test Plan}
\date{28 Apr 2010}

\begin{document}

\maketitle

\section*{Design Document}

The idea behind this project is to offer up a replacement for the default *nix shells (sh/bash/zsh/tcsh). The shell serves as a command line interpereter which executes userspace programs on behalf of the user. The must then be able to offer a command line prompt, fork new processes and offer at least rudimentary redirection capabilities. This project began as a simple shell written for CS201. Background, pipe and redirection logic were added recently, and some code was taken from both ``Modern Operating Systems`` by Tanenbaum and ``Advanced Programming in the UNIX environment``, by Stevens.

My shell will be broken into modules which are arranged as follows:

\begin{enumerate}
    \item \textbf{Main:} Primary application logic and main loop.
    \item \textbf{Input:} Command line interpereter.
    \item \textbf{Proc:} Fork and exec logic, as well as backgrounding.
    \item \textbf{Redir:} Logic for pipe and redirection logic.
    \item \textbf{List:} Circular Linked List data structure for history mechanism.
    \item \textbf{Util:} Common utility routines.  \item \textbf{Mysleep:} A simple test program which sleeps for a random number of seconds.
\end{enumerate}

Roughly, this shell will do the following:

Take an input string containing keywords and parse it into commands. The first word in an input string is considered to be either a builtin keyword, or the name of a binary file contained within the system.  It should read one line at a time (newline terminated). The shell will accept the builtin keywords ``x``, ``exit``, ``logout`` or an EOF (Control-D) to exit.  A non-builtin command will be executed using the command using fork, execvp, and waitpid. The shell will fork a child, let the child call exec, and wait for the child to terminate. Upon initialization, a handler is declared for the ``SIGCHLD`` signal which watches for zombie processes and prevents them from running amok if the parent dies first. The following builtin commands are supported:
\begin{enumerate}
	\item \textbf{myshell} : Executes a line using the system command. Used only for manual testing and comparison.
	\item \textbf{history} : List all commands contained in history, along with an index.
	\item \textbf{!} : Execute a history command, based on its index. 
	\item \textbf{x, exit, logout} : Exit the shell
	\item \textbf{cd, chdir} : Change the shell directory.
	\item \textbf{$<$} : Redirects output from a single command to an output file.
	\item \textbf{$|$} : Redirects output from one command to a second command.
	\item \textbf{\&} : Executes a single process in the background without waiting.
\end{enumerate}

\section*{Strengths and Weaknesses}
    \subsection*{Strengths}
        Modular design.
        History mechanism.
        Directory change implemented.
        Pipe implemented using popen() allows for multiple pipes to be used.
    \subsection*{Weaknesses}
        No combinations of redirection, background and pipe have been implemented. These utilities may only be used one at a time. The background operator supports only one process at a time. The pipe and redirection operators support only two processes.
        Incomplete Redirection logic. The ``$<$`` operator was the only operator implemented for this shell. 

\section*{Functions, by module}
    \subsection*{1: Main}
    \begin{itemize}
        \item \textbf{main()}
            Main loop.
        \item \textbf{prettyprompt()}
            Print a prompt for the user.
        \item \textbf{getinput()}
            Get a raw line of input.
        \item \textbf{initialize()}
            Clear the screen for the user.
        \item \textbf{cleanup()}
            Clear the screen for the user.
    \end{itemize}

    \subsection*{2: Input}
    \begin{itemize}
        \item \textbf{handleinput()}
            Test for builtins.
    \end{itemize}

    \subsection*{3: Process}
    \begin{itemize}
        \item \textbf{havechildren()}
            Build a proper command string.
            Generate a child process. 
            Wait for child, unless a background character has been inserted.
        \item \textbf{ordinaryfork()}
            Fork a single process. Also does background handling.
        \item \textbf{redirectionfork()}
            Fork a redirected process.
        \item \textbf{pipefork()}
            Fork a piped process.
    \end{itemize}


    \subsection*{4: Redirection}
    \begin{itemize}
        \item \textbf{backgroundtest()}
            Check the input string to see if the command will be backgrounded.
        \item \textbf{operatortest()}
            Check the input string to see whether there will be piping or redirection.
    \end{itemize}

    \subsection*{5: List}
    \begin{itemize}
        \item \textbf{initlist()}
            A simple "constructor" which allocates memory.
        \item \textbf{addstring()}
            Add a single string to the list.
        \item \textbf{prinstrings()}
            Print all strings in the list.
        \item \textbf{printcmd()}
            Print a single string in the list, from its index.
        \item \textbf{getcmd()}
            Execute a command from the list based on its index.
        \item \textbf{destructlist()}
            Free all dynamic memory in the list.
    \end{itemize}

    \subsection*{6: Utility Functions}
    \begin{itemize}
        \item \textbf{tokenize()}
            Break a string into multiple null delimited tokens.
        \item \textbf{chomp()}
            Remove trailing newline characters from char arrays.
        \item \textbf{buildmulticmd()}
            Create an array of commands from an input string.
    \end{itemize}

    \subsection*{7: Mysleep}
    \begin{itemize}
        \item \textbf{main()}
            Sleeps for a random number of seconds.
    \end{itemize}


\section*{Test Plan}
    \subsection*{Test mechanism}
        Testing for this shell was done by piping input from a text file directly to the shell. Output is located in the file ``typescript`` and the test commands themselves are located in a file called ``qsh\_tests``, both packaged with the program source. A discussion of the implementation of the commands used can be found above. The commands used for testing are listed as follows: \\

.
\\ls -la\\
df\\
ls\\
cd ..\\
cd A3\\
chdir ..\\
history\\
chdir A3\\
ls -la\\
df\\
history\\
1 2 3 4 5 6 7 8 9 0 1 2 3 4 5 6 7 8 9 0 1 2 3 4 5 6 7 8 9 0 1 2 3 4 5 6 7 8 9 0 1 2 3 4 5 6 7 8 9 0 1 2 3 4 5 6 7 8 9 0 1 2 3 4 5 6 7 8 9 0 1 2 3 4 5 6 7 8 9 0 \\
history\\
history\\
ls $|$ grep c\\
ls $>$ ls\_output\\
mysleep \& \\ 
ls \\

\end{document}
